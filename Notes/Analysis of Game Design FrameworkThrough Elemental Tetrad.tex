\section{Categorizing Game Design Elements into Educational Game Design Fundamentals}

\subsection{Authors}
\begin{itemize}
    \item \textbf{Authors:} Mohamad Isa bin Ahyar
\end{itemize}

\subsection{Publication}
\begin{itemize}
    \item \textbf{Date:} April 2022
\end{itemize}

\subsection{Relevant to My Research}
Yes

\subsection{Aim}
better explain and analyze the Elemental Tetrad game design framework, and give examples

\subsection{Key Focus Areas}
The paper focuses on explainaing the Elemental Tetrad, which consists of four key components of a game: mechanics, story, aesthetics, and technology. The paper also provides examples of how the Elemental Tetrad can be applied to game design.
\\
the paper discuss the relation between the components
\begin{itemize}
    \item Aesthetics, which is described by the story and mechanics, and is supported by technology, and is about the atmosphere and the feel of the game.
    \item Mechanics, which is justified by the story, presented by the aesthetics, and is achiveable through technology, is about the rules and the gameplay.
    \item Story, which creates the mechanics, and is presented by the aesthetics, and is supported by technology, is about the narrative and the plot.
    \item Technology, must meet the need of the story by supporting the mechanics and aesthetics, is about the platform and the tools.
\end{itemize}

\subsection{Gaps Addressed}
\begin{itemize}
    \item The paper addresses the gap that is analysis of the Elemental Tetrad game design framework
\end{itemize}

\subsection{Findings}
\begin{itemize}
    \item The Elemental Tetrad is a model used in many game in the modern day.
\end{itemize}


\subsection{Future Work}
Not mentioned    

\section{Notes}
\begin{itemize}
\item the Elemental Tetrad is a model that describes the four key components of a game: mechanics, story, aesthetics, and technology.

\end{itemize}
