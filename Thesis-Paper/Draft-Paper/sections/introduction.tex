\chapter{Introduction}
\label{chap:intro}
Gamification, the practice of integrating game-like elements into non-game settings, has become increasingly influential in education over the past decade. Its potential to boost student motivation, foster engagement, and improve academic performance has made it a go-to strategy for educators worldwide. By incorporating features such as points, badges, leaderboards, and challenges, gamification transforms the learning experience into an interactive and rewarding process, inspiring students to take an active role in their education.

Research consistently highlights the benefits of gamification in educational contexts. For instance, Lara-Cabrera et al. (2023) found that gamified techniques, such as awarding 3D-printed badges, not only enhanced academic performance but also reduced dropout rates, especially in STEM programs at the higher education level \cite{lara2023badges}. Similarly, Jack et al. (2024) examined gamification within flipped classrooms, demonstrating its ability to significantly increase student engagement and motivation by promoting active learning \cite{jack2024gamification}. Other studies, such as one published in Sustainability (2022), revealed that gamified learning environments can alleviate stress and anxiety, allowing students to concentrate better and perform more effectively \cite{sustainability2022gamification}. Game-Based Learning (GBL) has also proven effective in improving student outcomes. For example, Fernández-Alemán et al. (2024) showed that GBL enhances motivation through immediate feedback, clear objectives, and a sense of achievement \cite{fernando2024}. These studies collectively underscore gamification’s transformative potential in education, highlighting its ability to create meaningful and enjoyable learning experiences that extend beyond conventional teaching methods. Moreover, advancements in personalized gamification have been explored. Reza et al. (2021) proposed an automated system for generating customized challenges based on players’ preferences, game progress, and performance, addressing existing limitations in gamified systems \cite{khoshkangini2021}.

Despite its advantages, gamification is still a developing field with notable challenges. While its effectiveness has been widely documented, one critical limitation is the lack of accessible tools that enable educators to design gamified learning experiences without requiring technical expertise in programming or game design. Many current systems necessitate specialized knowledge, creating barriers for instructors who may lack the resources or skills to implement gamified strategies independently. Bridging this gap by developing user-friendly tools could empower a broader range of educators to leverage gamification, ultimately fostering more personalized and engaging learning environments.