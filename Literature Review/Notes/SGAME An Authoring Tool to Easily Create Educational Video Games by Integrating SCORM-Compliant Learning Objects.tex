\section{SGAME: An Authoring Tool to Easily Create Educational Video Games by Integrating SCORM-Compliant Learning Objects}

\subsection{Authors}
\begin{itemize}
    \item \textbf{Authors:} ALDO GORDILLO, ENRIQUE BARRA, AND JUAN QUEMADA 
\end{itemize}

\subsection{Publication}
\begin{itemize}
    \item \textbf{Publication:} Digital Object Identifier 10.1109/ACCESS.2021.3111513
    \item \textbf{Date:} 9 September 2021
\end{itemize}

\subsection{Relevant to My Research}
Yes

\subsection{Aim}
This article presents a teacher-oriented authoring tool for educational games called SGAME. It explores teachers' perceptions of the tool and evaluates the effect of the educational video games it helps create on students' perceptions and learning outcomes.

\subsection{Key Focus Areas}
Three evaluation instruments were used:
\begin{itemize}
    \item A questionnaire to gather teachers' perceptions of the SGAME authoring tool.
    \item A questionnaire to collect students' perceptions of a game created with SGAME.
    \item Pre- and post-tests to assess students' learning gains from playing the game.
\end{itemize}

A total of 201 teachers and 79 students participated in the evaluation. The results show that SGAME is a user-friendly tool for creating effective and motivating educational video games.

\subsection{Gaps Addressed}
The use of game-based learning in educational settings is limited by the lack of authoring tools that allow teachers to easily create and adapt educational video games to their needs and contexts.

\subsection{Findings}
\begin{itemize}
    \item \textbf{Positive Impact}: SGAME increases student motivation and learning outcomes.
    \item \textbf{Ease of Use}: Teachers found SGAME easy to use.
    \item \textbf{Effective Tool}: Case study results showed significant learning gains.
    \item \textbf{Teacher Perceptions}: SGAME was especially beneficial in primary, secondary, and special-needs education.
    \item \textbf{Areas for Improvement}: There is a need for more game templates, resources, and sequencing options.
\end{itemize}

\subsection{Limitations}
\begin{itemize}
    \item \textbf{Volunteer Bias}: The evaluation used convenience sampling, meaning only teachers who volunteered participated, which may introduce bias.
    \item \textbf{Subjective Usability Assessment}: The usability of SGAME was evaluated solely based on teachers' perceptions, lacking formal usability testing.
    \item \textbf{Limited Templates}: SGAME currently lacks a wide variety of game templates, limiting its flexibility.
\end{itemize}

\subsection{Future Work}
\begin{itemize}
    \item \textbf{Expand Game Templates}: Increasing the variety of game templates, particularly for mobile and tablet devices.
    \item \textbf{Enhanced Resources}: Providing more tutorials and expanding the user manual to help teachers use SGAME more effectively.
    \item \textbf{Advanced Options}: Adding more sequencing options and grading settings to improve the customization of games.
    \item \textbf{Wider Testing}: Conducting more studies in diverse educational settings to validate SGAME's long-term effectiveness.
\end{itemize}

\section{Notes}
Several popular game engines such as Unity, Unreal Engine, CryEngine, ImpactJS, and Phaser can be used to develop educational video games, but they require strong programming skills, making them unsuitable for most educators. Conversely, teacher-oriented authoring tools like SGAME aim to simplify the creation of educational video games for non-technical users.

\subsection{Similar Tools to SGAME}
Several teacher-oriented tools exist for creating educational games, such as:
\begin{itemize}
    \item \textbf{eAdventure} (Java-based, no longer in use).
    \item \textbf{uAdventure} (Successor of eAdventure, built on Unity) - \url{https://github.com/e-ucm/uAdventure?tab=readme-ov-file}.
    \item \textbf{StoryTec} (Story-based game creation) - \url{https://ieeexplore.ieee.org/document/4688056}.
    \item \textbf{IOLAOS} (Game templates customized by teachers) - \url{https://www.researchgate.net/publication/275645732_Ludic_Educational_Game_Creation_Tool_Teaching_Schoolers_Road_Safety}.
    \item \textbf{EMERGO} (Scenario-based game creation) - \url{https://www.ou.nl/emergo}.
\end{itemize}

Other notable tools include Genial.ly for creating educational escape rooms and ARLEARN for developing educational video games, though neither has undergone significant evaluation for their educational utility.
