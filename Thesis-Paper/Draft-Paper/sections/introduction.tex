\chapter{Introduction}\label{chap:intro}

\section{Motivation}
The generation of games started as game developers created games for entertainment purposes. However, the idea of using games for educational purposes has been around for a long time. Educational games are designed to teach players about specific subjects, reinforce concepts, facilitate skill development, or help them understand historical events or cultures through gameplay \cite{crawford1982art}. Educational games can be used in various educational settings, such as schools, universities, and training programs, to engage students and enhance their learning experience. By combining educational content with game mechanics, educators can create interactive and engaging learning activities that motivate students to learn and retain information effectively. Developing a game is a process that requires many skills and resources, including programming knowledge, game design expertise, and multimedia content creation. This can be a significant barrier for educators who lack the technical skills or resources to create their own educational games. To address this challenge, game-based learning platforms have been developed to provide educators with the tools and resources they need to design and implement educational games in their classrooms. Gamification is a proven means of education that has been show to improve the morale of the user through challenges, rewards, and competition. 

\section{Aim of the Study}
Gamification and the incorporation of game-design elements in non-game contexts has gained considerable momentum in educational settings. The potential it has to raise student motivation, engagement, and improve academic performance has positioned it as one of the most well-known strategies among educators. By incorporating elements of points, badges, leaderboards, and challenges, gamification converts the monotony of learning into an enjoyable process that ultimately renders students proactive in learning.

The issue arises when an instructor needs to create a gamified learning experience for the students, and lacks the skills to create a game on the subject that is being taught. The instructor would need to learn game design and programming to be able to create such game, and even assuming that the instructor in specialized within the field of game design and programming, the time and effort needed to generate a game would be massive and impractical. This is where the tool presented in this paper comes in to play, the tool allows the instructor to generate a game without the need of learning programming or game design, and only needs the content that would be used within the game itself. The tool is designed to be educational and ensures gamified elements are present within the game. 

\section{Thesis Outline}
the work presented in this paper is divided into five chapters. Chapter 2 provides an overview of the related work in the field of game-based learning and gamification in education. Chapter 3 describes the design and implementation of the proposed tool, including the system architecture, features, and functionality. Chapter 4 presents the experimental setup and results of the usability study and focus group session conducted with educators to evaluate the tool. Chapter 5 discusses the limitations of the proposed tool and the implications for future research as well as the conclusion of the study.