\chapter{Background}\label{chap:background}
The generation of games started as game developers created games for entertainment purposes. However, the idea of using games for educational purposes has been around for a long time. Educational games are designed to teach players about specific subjects, reinforce concepts, facilitate skill development, or help them understand historical events or cultures through gameplay \cite{crawford1982art}. Educational games can be used in various educational settings, such as schools, universities, and training programs, to engage students and enhance their learning experience. By combining educational content with game mechanics, educators can create interactive and engaging learning activities that motivate students to learn and retain information effectively. Developing a game is a process that requires many skills and resources, including programming knowledge, game design expertise, and multimedia content creation. This can be a significant barrier for educators who lack the technical skills or resources to create their own educational games. To address this challenge, game-based learning platforms have been developed to provide educators with the tools and resources they need to design and implement educational games in their classrooms. Gamification is a proven means of education that has been show to improve the morale of the user through challenges, rewards, and competition. 

\section{Game-Based Learning}
Game-based learning (GBL) is an educational approach that uses games to teach students about specific subjects, concepts, or skills. GBL combines educational content with game mechanics to create interactive and engaging learning experiences that motivate students to learn and retain information effectively. By incorporating elements such as challenges, rewards, competition, and feedback, GBL can enhance student engagement, motivation, and learning outcomes. GBL can be used in various educational settings, such as schools, universities, and training programs, to facilitate skill development, reinforce concepts, and promote active learning. For example, a literature review conducted by Fernando et al. (2024) found that GBL can enhance student motivation through immediate feedback, clear objectives, and a sense of achievement \cite{fernando2024}. These studies highlight the potential of GBL to transform the learning experience and create meaningful and enjoyable learning activities that extend beyond traditional teaching methods.

\section{Gamification}
Gamification is the practice of integrating game-like elements into non-game settings to motivate and engage users. By incorporating features such as points, badges, leaderboards, challenges, and rewards, gamification transforms the user experience into an interactive and rewarding process that encourages participation and achievement. Gamification has been widely used in various domains, such as marketing, health, fitness, and education, to motivate users, drive behavior change, and enhance engagement. In education, gamification has gained popularity as a means of boosting student motivation, fostering engagement, and improving learning outcomes. Research has shown that gamification can enhance student performance, increase motivation, and reduce dropout rates across different educational levels and subjects \cite{lara2023badges}. For example, a study conducted by Lara-Cabrera et al. (2023) found that gamified techniques, such as awarding 3D-printed badges, can enhance academic performance and reduce dropout rates, especially in STEM programs at the higher education level \cite{lara2023badges}. Another study published in the Journal of Educational Technology (2022) demonstrated that gamification can increase student engagement and motivation in flipped classrooms by promoting active learning and collaboration \cite{jack2024gamification}. These studies highlight the potential of gamification to transform the learning experience and create personalized and engaging learning activities that motivate students to take an active role in their education.
