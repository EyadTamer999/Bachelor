\section{Use of Gamification and Game-Based Learning in Educating Generation Alpha: A Systematic Literature Review}

\subsection{Authors}
\begin{itemize}
    \item \textbf{Authors:} Pumudu A. Fernando
    \item  \textbf{Authors:} Salinda Premadasa
\end{itemize}

\subsection{Publication}
\begin{itemize}
    \item \textbf{Publication:} Educational Technology \& Society 27(2):114-132
    \item \textbf{Date:} April 2024 
\end{itemize}

\subsection{Relevant to the My Research}
Yes

\subsection{Aim}
The study aims to explore the current state of gamification and game-based learning adoption for primary education students, using recent peer-reviewed research. Through a systematic mapping design, the reviewed papers are categorized and analyzed based on attributes like:
\begin{itemize}
    \item Type of gamification and game mechanics.
    \item Evaluation context.
    \item Experimental outcomes.
    \item Academic subjects.
    \item Types of applications involved.
\end{itemize}

\subsection{Key Focus Areas}
\begin{itemize}
    \item Type of gamification and game mechanics used.
    \item Focus on Generation Alpha students (primary education).
    \item Educational context and subjects involved.
    \item Experimental outcomes and effectiveness.
\end{itemize}

\subsection{Gaps Addressed}
\begin{itemize}
    \item Limited to primary education students, excluding secondary and tertiary levels.
    \item Gen Alpha students get bored due to their changing interests.
\end{itemize}

\subsection{Future Work}

\subsection{Notes} 
The paper defines "Gamification" as "the use of game design elements in non-game contexts." 
It also mentions that components of games, such as points, badges, and challenges, 
are employed in gamification, not to build a game, 
but to motivate and enhance the learner's experience and increase engagement.
The paper cites and explains the two diffrent types of gamification:
\begin{itemize}
    \item \textbf{Structural Gamification:} This type of gamification involves the use of game elements like points, badges, and leaderboards to motivate learners.
    \item \textbf{Content Gamification:} This type of gamification involves the use of game elements like stories, characters, and challenges to engage learners.
\end{itemize}
so in more simple words structural gamification is about the rewards and content gamification is about the story and characters.
There also exists Game-Based Learning (GBL) which is the use of games to teach students, 
as the game is mapped to the learning objectives.
Structural and content gamification both are means to motivate learners, however GBL is a means to teach students.