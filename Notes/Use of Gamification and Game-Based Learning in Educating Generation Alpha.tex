\section{Use of Gamification and Game-Based Learning in Educating Generation Alpha: A Systematic Literature Review}

\subsection{Authors}
\begin{itemize}
    \item \textbf{Authors:} Pumudu A. Fernando
    \item  \textbf{Authors:} Salinda Premadasa
\end{itemize}

\subsection{Publication}
\begin{itemize}
    \item \textbf{Publication:} Educational Technology \& Society 27(2):114-132
    \item \textbf{Date:} April 2024 
\end{itemize}

\subsection{Relevant to the My Research}
Yes

\subsection{Aim}
The study aims to explore the current state of gamification and game-based learning adoption for primary education students, using recent peer-reviewed research. Through a systematic mapping design, the reviewed papers are categorized and analyzed based on attributes like:
\begin{itemize}
    \item Type of gamification and game mechanics.
    \item Evaluation context.
    \item Experimental outcomes.
    \item Academic subjects.
    \item Types of applications involved.
\end{itemize}

\subsection{Key Focus Areas}
\begin{itemize}
    \item Type of gamification and game mechanics used.
    \item Focus on Generation Alpha students (primary education).
    \item Educational context and subjects involved.
    \item Experimental outcomes and effectiveness.
\end{itemize}

\subsection{Gaps Addressed}
\begin{itemize}
    \item Limited to primary education students, excluding secondary and tertiary levels.
    \item Gen Alpha students get bored due to their changing interests.
\end{itemize}

\subsection{Future Work}

\begin{itemize}
    \item \textbf{Observation of Realistic Behavioral Attributes of Generation Alpha:}
    Further empirical and qualitative studies are necessary to confirm the actual learning preferences and behavioral patterns of Generation Alpha, as current research largely relies on assumptions derived from previous generations.

    \item \textbf{Need for Novel Gamification Mechanics:}
    New, more engaging, and innovative game mechanics are required to go beyond the commonly used Points, Badges, and Leaderboards (PBL) framework to maintain the attention of Generation Alpha learners.
    Future research should explore emerging designs such as augmented reality, gamified e-books, interactive avatars, and narrative-driven role-playing games. It is also important to integrate mobile-friendly and online social communication features, such as chatbots and instant feedback, to align with Generation Alpha's preference for digital environments.

    \item \textbf{Need for Adaptive Gamification-Based Learning Strategies:}
    Further studies are needed to explore adaptive gamification strategies that account for individual learning styles, cognitive abilities, emotions, and skill levels. Research into machine learning and deep learning techniques is critical for predicting student performance and personalizing the in-game learning experience. Additionally, more refined adaptive algorithms should be developed to improve the effectiveness of adaptive games over traditional instructional methods.

    \item \textbf{Need for Long-Term Studies with Diverse Learner Samples:}
    Long-term and large-scale experiments are essential for better understanding the sustained effects of gamified learning and its long-term psychological impact. Studies should also consider factors such as gender, skill level, socio-economic background, and prior knowledge when selecting learners for evaluating the impact of gamification on education.
\end{itemize}

\subsection{Notes} 
The paper defines "Gamification" as "the use of game design elements in non-game contexts." 
It also mentions that components of games, such as points, badges, and challenges, 
are employed in gamification, not to build a game, 
but to motivate and enhance the learner's experience and increase engagement.
The paper cites and explains the two diffrent types of gamification:
\begin{itemize}
    \item \textbf{Structural Gamification:} This type of gamification involves the use of game elements like points, badges, and leaderboards to motivate learners.
    \item \textbf{Content Gamification:} This type of gamification involves the use of game elements like stories, characters, and challenges to engage learners.
\end{itemize}
so in more simple words structural gamification is about the rewards and content gamification is about the story and characters.
There also exists Game-Based Learning (GBL) which is the use of games to teach students, 
as the game is mapped to the learning objectives.
Structural and content gamification both are means to motivate learners, however GBL is a means to teach students.
\\
In the gamification elements and mechanics, the paper mentions the following:
    the use of combinations of of storylines, narratives, and avatars appears to be a popular choice among these studies, while the traditional
    triad of game elements - points, badges, and leaderboards (PBL) - are used less frequently and in fewer
    combinations.
\\
Most studies on gamified learning have found positive effects on learners' motivation, attention, performance, and engagement, though a few have shown mixed results. However, many of these findings rely on qualitative comments and surveys rather than empirical data, and some studies lack sufficient information about the experiment context, duration, and sample selection, making it difficult to conduct a meta-analysis. Most research has focused on using gamification in Mathematics for primary school children, with mobile educational games and apps being seen as ideal, particularly due to Generation Alpha's increased screen time and use of mobile devices.
