\documentclass{article}
\usepackage{hyperref}
\usepackage{amsmath}
\usepackage[utf8]{inputenc}

\title{Abstracts Analysis Notes}
\author{Eyad Hassan}
\date{\today}

\begin{document}

\maketitle

\section{Introduction}
This document contains an analysis of papers analyzed

\newpage
\section{Use of Gamification and Game-Based Learning in Educating Generation Alpha: A Systematic Literature Review}

\subsection{Authors}
\begin{itemize}
    \item \textbf{Authors:} Pumudu A. Fernando, Salinda Premadasa
\end{itemize}

\subsection{Publication}
\begin{itemize}
    \item \textbf{Publication:} Educational Technology \& Society 27(2):114-132
    \item \textbf{Date:} April 2024 
\end{itemize}

\subsection{Relevant to the My Research}
Yes

\subsection{Aim}
The study aims to explore the current state of gamification and game-based learning adoption for primary education students, using recent peer-reviewed research. Through a systematic mapping design, the reviewed papers are categorized and analyzed based on attributes like:
\begin{itemize}
    \item Type of gamification and game mechanics.
    \item Evaluation context.
    \item Experimental outcomes.
    \item Academic subjects.
    \item Types of applications involved.
\end{itemize}

\subsection{Key Focus Areas}
\begin{itemize}
    \item Type of gamification and game mechanics used.
    \item Focus on Generation Alpha students (primary education).
    \item Educational context and subjects involved.
    \item Experimental outcomes and effectiveness.
\end{itemize}

\subsection{Gaps Addressed}
\begin{itemize}
    \item Limited to primary education students, excluding secondary and tertiary levels.
    \item Gen Alpha students get bored due to their changing interests.
\end{itemize}

\subsection{Findings}
\begin{itemize}
    \item Gamification and game-based learning (GBL) showed positive effects on student engagement and motivation, especially in primary school subjects such as mathematics. Game elements like quizzes, storylines, and avatars were effective in boosting participation.
    
    \item Generation Alpha students generally preferred interactive and gamified learning, although some negative reactions were noted for leaderboards, which could demotivate lower-performing students. Gender differences were observed in student responses to game elements.
    
    \item Gamification strategies, particularly those with adaptive elements, improved student performance, collaboration, and social skills. These methods align well with Generation Alpha’s preference for technology-driven, interactive learning.
\end{itemize}

\subsection{Future Work}
\begin{itemize}
    \item \textbf{Observation of Realistic Behavioral Attributes of Generation Alpha:}
    Further empirical and qualitative studies are necessary to confirm the actual learning preferences and behavioral patterns of Generation Alpha, as current research largely relies on assumptions derived from previous generations.

    \item \textbf{Need for Novel Gamification Mechanics:}
    New, more engaging, and innovative game mechanics are required to go beyond the commonly used Points, Badges, and Leaderboards (PBL) framework to maintain the attention of Generation Alpha learners.
    Future research should explore emerging designs such as augmented reality, gamified e-books, interactive avatars, and narrative-driven role-playing games. It is also important to integrate mobile-friendly and online social communication features, such as chatbots and instant feedback, to align with Generation Alpha's preference for digital environments.

    \item \textbf{Need for Adaptive Gamification-Based Learning Strategies:}
    Further studies are needed to explore adaptive gamification strategies that account for individual learning styles, cognitive abilities, emotions, and skill levels. Research into machine learning and deep learning techniques is critical for predicting student performance and personalizing the in-game learning experience. Additionally, more refined adaptive algorithms should be developed to improve the effectiveness of adaptive games over traditional instructional methods.

    \item \textbf{Need for Long-Term Studies with Diverse Learner Samples:}
    Long-term and large-scale experiments are essential for better understanding the sustained effects of gamified learning and its long-term psychological impact. Studies should also consider factors such as gender, skill level, socio-economic background, and prior knowledge when selecting learners for evaluating the impact of gamification on education.
\end{itemize}

\subsection{Notes} 
The paper defines "Gamification" as "the use of game design elements in non-game contexts." 
It also mentions that components of games, such as points, badges, and challenges, 
are employed in gamification, not to build a game, 
but to motivate and enhance the learner's experience and increase engagement.
The paper cites and explains the two diffrent types of gamification:
\begin{itemize}
    \item \textbf{Structural Gamification:} This type of gamification involves the use of game elements like points, badges, and leaderboards to motivate learners.
    \item \textbf{Content Gamification:} This type of gamification involves the use of game elements like stories, characters, and challenges to engage learners.
\end{itemize}
so in more simple words structural gamification is about the rewards and content gamification is about the story and characters.
There also exists Game-Based Learning (GBL) which is the use of games to teach students, 
as the game is mapped to the learning objectives.
Structural and content gamification both are means to motivate learners, however GBL is a means to teach students.
\\
In the gamification elements and mechanics, the paper mentions the following:
    the use of combinations of of storylines, narratives, and avatars appears to be a popular choice among these studies, while the traditional
    triad of game elements - points, badges, and leaderboards (PBL) - are used less frequently and in fewer
    combinations.
\\
Most studies on gamified learning have found positive effects on learners' motivation, attention, performance, and engagement, though a few have shown mixed results. However, many of these findings rely on qualitative comments and surveys rather than empirical data, and some studies lack sufficient information about the experiment context, duration, and sample selection, making it difficult to conduct a meta-analysis. Most research has focused on using gamification in Mathematics for primary school children, with mobile educational games and apps being seen as ideal, particularly due to Generation Alpha's increased screen time and use of mobile devices.


\newpage
\section{Generative AI for Customizable Learning Experiences}

\subsection{Authors}
\begin{itemize}
    \item \textbf{Authors:} Ivica Pesovski, Ricardo Santos, Roberto Henriques, Vladimir Trajkovik
\end{itemize}

\subsection{Publication}
\begin{itemize}
    \item \textbf{Publication:} Sustainability, Volume 16, Article 3034
    \item \textbf{Date:} April 2024
    \item \textbf{DOI:} \href{https://doi.org/10.3390/su16073034}{10.3390/su16073034}
\end{itemize}

\subsection{Relevant to My Research}
Loosely relevant

\subsection{Aim}
The paper proposes an affordable and sustainable approach to personalizing learning materials, developing a tool integrated into an LMS.

\subsection{Key Focus Areas}
\begin{itemize}
    \item Generative AI and Personalized Learning.
    \item Tool integrated into LMS for generating materials based on learning outcomes.
    \item Learning materials in three formats: traditional, pop-culture-inspired (e.g., Batman, Wednesday Addams).
    \item Assessment via multiple-choice questions.
    \item Experiment with 20 software engineering students.
    \item Positive findings: engagement and increased study time.
\end{itemize}

\subsection{Gaps Addressed}
\begin{itemize}
    \item Lack of empirical studies on LLMs in classrooms.
    \item Limited research on AI-generated lesson creation.
    \item Unclear effectiveness of virtual AI instructors.
\end{itemize}

\subsection{Future Work}
\begin{itemize}
    \item Further study on integrating LLMs into more educational contexts.
    \item Long-term research on the reception of AI-generated content.
\end{itemize}

\subsection{Notes}
The paper emphasizes that generative AI can engage students through personalized, multi-format content that caters to various learning styles.


\newpage
\section{Art of Computer Game Design}

\subsection{Authors}
\begin{itemize}
    \item \textbf{Author:} Chris Crawford
\end{itemize}

\subsection{Publication}
\begin{itemize}
    \item \textbf{Publication:} Washington State University
    \item \textbf{Date:} Originally published in 1982, with an electronic version in 1997
    \item \textbf{DOI:} Not available; however, the electronic version is accessible at \href{http://www.vancouver.wsu.edu/fac/peabody/game-book/Coverpage.html}{Washington State University Vancouver}
\end{itemize}

\subsection{Relevant to My Research}
Highly relevant for understanding the fundamental principles of game design and the role of games as an art form.

\subsection{Aim}
The book explores the concept that computer games are a new art form, focusing on how games evoke emotion through fantasy, and discusses various aspects of game design, including representation, interaction, conflict, and safety.

\subsection{Key Focus Areas}
\begin{itemize}
    \item Definition of games and their fundamental components: representation, interaction, conflict, and safety.
    \item The role of fantasy and participation in games, making them an interactive art form.
    \item A comparison of games with other art forms like music and literature, emphasizing the participatory nature of games.
    \item Game design principles and techniques.
    \item The importance of emotional engagement in games.
    \item A taxonomy of computer games, including skill-and-action games, strategy games, and more.
\end{itemize}

\subsection{Gaps Addressed}
\begin{itemize}
    \item Lack of established theories on game design.
    \item Minimal focus on the artistic potential of computer games in the early 1980s.
    \item The book challenges the view that computer games are mere entertainment, emphasizing their capacity as an art form.
\end{itemize}

\subsection{Findings}
\begin{itemize}
    \item Games are an interactive form of art, allowing players to create their own experiences within the framework provided by the designer.
    \item The participatory nature of games offers a deeper emotional engagement compared to other art forms.
    \item Computer games, though still developing as an art form in the early 1980s, hold vast potential for emotional and intellectual engagement.
    \item The success of game design depends on understanding the fundamental elements of games and creating experiences that resonate emotionally with players.
\end{itemize}

\subsection{Future Work}
\begin{itemize}
    \item Further exploration of games as an art form.
    \item Development of more sophisticated game design theories and methodologies.
    \item Expansion of the taxonomy of computer games as the industry evolves.
\end{itemize}

\subsection{Notes}
This foundational book in the field of game design emphasizes the artistic and participatory aspects of games, proposing that computer games have the potential to evoke deep emotional responses through interactive fantasy worlds. The text provides a basis for understanding the core principles of game design and their artistic potential.


\newpage
\section{Automatic Generation and Recommendation of Personalized Challenges for Gamification}

\subsection{Authors}
\begin{itemize}
    \item \textbf{Authors:} Reza Khoshkangini, Giuseppe Valetto, Annapaola Marconi, Marco Pistore
\end{itemize}

\subsection{Publication}
\begin{itemize}
    \item \textbf{Publication:} Springer Nature
    \item \textbf{Date:} 24 May 2020
\end{itemize}

\subsection{Relevant to My Research}
Yes

\subsection{Aim}
The study aims to address common gamification limitations by proposing an automatic system for generating personalized and contextualized challenges based on player preferences, game status, and performance.

\begin{itemize}
    \item Overcoming boredom and frustration caused by generic game elements like points, badges, and leaderboards.
    \item Automatically generating challenges that are tailored to individual players.
    \item Comparative evaluation between manually designed challenges and those generated automatically.
\end{itemize}

\subsection{Key Focus Areas}
\begin{itemize}
    \item Use of Procedural Content Generation (PCG) for challenges.
    \item Player-specific challenge personalization and contextualization.
    \item Efficiency of automatically generated challenges in gamification.
\end{itemize}

\subsection{Gaps Addressed}
\begin{itemize}
    \item The lack of personalized game elements in many gamification systems.
    \item Reliance on static game elements, which fail to engage players long-term.
\end{itemize}

\subsection{Findings}
\begin{itemize}
    \item Automatically generated challenges showed comparable, if not superior, results in keeping players engaged compared to manually designed challenges.
    \item Personalized challenges improved user engagement and retention in a 12-week urban mobility experiment.
    \item Dynamic challenge assignment aligned with player performance and game objectives can reduce the effort required by human game designers.
\end{itemize}

\subsection{Limitations}
\begin{itemize}
    \item period of the experiment was relatively short (3 weeks).
    \item it does not include
    players’ characteristics and physiological signals, which are widely used in digital
    games for advanced personalization and to increase players’ engagement
    \item player's elevation status was not considered in the challenges.
\end{itemize}

\subsection{Future Work}
\begin{itemize}
    \item \textbf{Expansion of the System to Other Domains:} The system could be adapted to other gamification contexts beyond urban mobility.
    
    \item \textbf{Further Research on Procedural Content Generation in Gamification:} More studies on PCG for challenge personalization in different application domains.
    
    \item \textbf{Integration of Machine Learning Techniques:} Explore the use of ML to enhance personalization and adapt challenges in real-time.
    
    \item \textbf{Long-Term Impact Assessment:} Future work could focus on assessing the long-term impact of personalized challenges on user behavior and sustained engagement.
    
\end{itemize}

\section{Notes}
\subsection{Research Questions}

\begin{itemize}
    \item \textbf{RQ1: Player Acceptance:} How does the player acceptance rate for automatically generated challenges compare to the acceptance rate of manually assigned challenges?
    \item \textbf{RQ2: Challenge Impact:} How does the improvement recorded on the target goal for automatically generated challenges compare to the improvement recorded for manually assigned challenges?
    \item \textbf{RQ3: Reward Efficiency:} How do the rewards computed for automatically generated challenges compare to those of manually assigned challenges?
\end{itemize}

\subsection{Trento Play\&Go: A Gamification Campaign}

Trento Play\&Go was a large-scale, long-running open-field gamification campaign that lasted 12 weeks (from September 10 to December 2, 2016). The game targeted residents of Trento, Italy, and commuters from the surrounding Trentino province. Participants used the Viaggia Play\&Go mobile app to plan and track journeys, check their game status, share results, and receive notifications.

\subsection{The Challenge Model}

A challenge model is a system for creating personalized tasks for players. It consists of the following components:

\begin{itemize}
    \item \textbf{Player (P)}: The individual playing the game.
    \item \textbf{Goal (G)}: The task or objective to be completed.
    \item \textbf{Constraint (C)}: Conditions for reaching the goal (e.g., deadline).
    \item \textbf{Difficulty (D)}: The level of challenge for the player.
    \item \textbf{Reward (R)}: The prize awarded for completing the challenge.
    \item \textbf{Weight (W)}: The importance of the challenge.
\end{itemize}

\subsection{Example Challenges}

\begin{itemize}
    \item \textbf{Example 1:} Increase walking distance by 10\% next week and earn 200 points (Green Leaves).
    \item \textbf{Example 2:} Increase train trips by 30\% next month and earn an additional 20 points per trip.
    \item \textbf{Example 3:} Complete at least 1 bike-sharing trip next week and earn 80 points.
\end{itemize}

\subsection{How the System Works}

\begin{enumerate}
    \item The **challenge generator** creates a set of possible challenges.
    \item The **challenge valuator** calculates difficulty and assigns rewards.
    \item The **filtering and sorting module** recommends challenges based on player profile, game history, and campaign objectives.
\end{enumerate}

\subsection{RQ1: Player Acceptance}

To evaluate RQ1, the success rates of players in completing automatically generated challenges (RS challenges) were compared to those assigned manually. The analysis focused on 82 RS players during weeks 10-12 of Trento Play\&Go, who received a total of 220 RS challenges. The completion rates of these players were contrasted with those of non-RS players categorized into four distinct groups:

\begin{itemize}
    \item \textbf{Group 1:} This group includes both RS and non-RS players.
    \item \textbf{Group 2:} A subset of Group 1, consisting solely of active players during weeks 10-12.
    \item \textbf{Group 3:} This group excludes top performers.
    \item \textbf{Group 4:} This group compares the performance of RS players on RS challenges versus challenges assigned by experts.
\end{itemize}

An equivalence test (TOST) was employed to determine if the completion rates were statistically similar. The results indicated the following:

\begin{itemize}
    \item RS players exhibited superior performance in Group 1; however, this group included a significant number of inactive players.
    \item In Group 2, non-RS players demonstrated slightly better performance, likely due to the presence of weekly champions.
    \item In Groups 3 and 4, RS and non-RS challenges were found to be statistically equivalent, suggesting no significant difference in player acceptance between the two types of challenges.
\end{itemize}

Therefore, it can be concluded that there is no significant difference in player acceptance between RS challenges and expert-assigned challenges.

\subsection{RQ2: Challenge Impact}

For RQ2, improvement during a challenge was measured relative to a player’s performance in the previous week using the following formula:

\begin{equation}
\text{Imp} = \frac{\text{counter} - \text{base}}{\text{base}}
\end{equation}

In this equation, \textit{counter} represents the current performance indicator value (ranging from $0$ to $+\infty$) and \textit{base} refers to the previous week’s value. Improvement ($\text{Imp}$) ranges from $\left[-1, +\infty\right]$, where $-1$ indicates no action towards the goal, and $0$ signifies no improvement.

The focus was placed on \textit{percentageIncrement} challenges, with 129 out of 164 categorized as such. These challenges were divided into those aimed at improving the number of trips and those focused on distance (Km). Comparisons between RS and non-RS challenges revealed that RS challenges generally resulted in greater improvements.

Quantitative analysis utilizing the area under the improvement curve (AUiC) demonstrated that RS challenges frequently yielded higher values, with significant differences in positive improvement highlighted through the Wilcoxon test (p-values of 0.0003194 for Km-related challenges and 6.549e-06 for trip-related challenges). Thus, RS challenges appear to facilitate greater improvement compared to manually assigned challenges.

\section{RQ3: Reward Efficiency}

To evaluate RQ3, the improvement achieved through various challenge types was correlated with the rewards allocated by the game for challenge completion. Improvement was characterized using the AUiC+ data in Table 3, concentrating on players who achieved improvement in their challenges, as non-RS challenges often resulted in negative total AUiC.

Given the differing sizes of player sets yielding improvement in RS versus non-RS scenarios, the data were normalized based on the number of players contributing to that improvement (as detailed in the "players improved" column of Table 3). The per capita reward attributed by the system per unit of AUiC+ (denoted as Reward\textsubscript{pc}) was computed using the formula:

\begin{equation}
\text{Reward}_{pc} = \frac{\text{Reward}_{tot}/\text{players}_{imp}}{\text{AUiC}^+}
\end{equation}

The data suggest that challenge proposals generated by the system are more cost-effective concerning rewards as incentives per unit of improvement. RS challenges consistently yield lower rewards per unit of improvement when compared to non-RS challenges. Specifically, the per capita reward attributed per unit of improvement in non-RS trip-based challenges is 1.91 times higher (561/293) than in the RS case, while in non-RS Km-based challenges, it is 2.3 times higher (782/340).

Moreover, the experiment indicates that players could enhance their performance even without completing the assigned challenges. For instance, one player increased his walking performance from 3 km in week "x" to approximately 4 km in week "x + 1," despite not completing the challenge aimed at improving to 5 km. This observation illustrates that the primary objective of gamified systems is to influence player behavior, with challenges serving as mechanisms to encourage such improvement. Therefore, further normalization (using Eq. 1) is essential for calculating the unit of improvement, irrespective of challenge completion.

In conclusion, challenges assigned by the system provide greater improvement for the same per capita reward, effectively achieving similar levels of improvement for lower rewards.


\section{Terminlology}
\begin{itemize}
    \item \textbf{Procedural Content Generation (PCG):} The automatic generation of game content, such as levels, challenges, and environments, using algorithms.
    \item \textbf{Recommendation System (RS):} A system that suggests items or actions to users based on their preferences, behavior, or context.
\end{itemize}

\newpage
\section{MDA: A Formal Approach to Game Design and Game Research}

\subsection{Authors}
\begin{itemize}
    \item \textbf{Authors:} Robin Hunicke, Marc LeBlanc, Robert Zubek
\end{itemize}

\subsection{Publication}
\begin{itemize}
    \item \textbf{Publication:} AAAI Workshop - Technical Report
    \item \textbf{Date:} January 2004
\end{itemize}

\subsection{Relevant to My Research}
Yes

\subsection{Aim}
The study aims to present the MDA framework, which provides a formal approach to game design and game research. The framework is based on three primary components: Mechanics, Dynamics, and Aesthetics.
\begin{itemize}
    \item \textbf{Mechanics:} The rules and systems that define the game's behavior.
    \item \textbf{Dynamics:} The run-time behavior of the mechanics acting on player inputs and each other.
    \item \textbf{Aesthetics:} The emotional responses evoked in players during gameplay.
\end{itemize}

\subsection{Key Focus Areas}
\begin{itemize}
    \item introduce the MDA framework as a formal approach to game design and research.
    \item discuss the three primary components of the MDA framework: Mechanics, Dynamics, and Aesthetics.
    \item provide examples of how the MDA framework can be applied to game design and research.
\end{itemize}

\subsection{Gaps Addressed}
\begin{itemize}
    \item provide a formal approach to game design and research.
    \item offer a structured framework for analyzing and designing games.
    \item emphasize the importance of aesthetics in game design.
\end{itemize}

\subsection{Gap in the Research}
\begin{itemize}
    \item The study does not provide empirical evidence to support the effectiveness of the MDA framework in game design and research.
    \item The study does not compare the MDA framework with other game design frameworks to highlight its unique features and advantages.
\end{itemize}

\subsection{Findings}
\begin{itemize}
    \item 
\end{itemize}


\subsection{Future Work}
\begin{itemize}
    \item Explore the integration of AI in the MDA framework to enhance the dynamics and aesthetics of games.
    \item Investigate the application of the MDA framework in emerging technologies such as VR, AR, and MR.
    \item Conduct empirical studies to validate the effectiveness of the MDA framework in game design and research.
\end{itemize}

\section{Notes}

\section{Terminlology}
\begin{itemize}
    \item \textbf{MDA Framework:} Mechanics, Dynamics, Aesthetics framework.
    \item \textbf{Mechanics:} The rules and systems that define the game's behavior.
    \item \textbf{Dynamics:} The run-time behavior of the mechanics acting on player inputs and each other.
    \item \textbf{Aesthetics:} The emotional responses evoked in players during gameplay.
\end{itemize}

\newpage
\section{Categorizing Game Design Elements into Educational Game Design Fundamentals}

\subsection{Authors}
\begin{itemize}
    \item \textbf{Authors:} Mifrah Ahmad
\end{itemize}

\subsection{Publication}
\begin{itemize}
    \item \textbf{Publication:}  10.5772/intechopen.89971
    \item \textbf{Date:} 20 November 2019
\end{itemize}

\subsection{Relevant to My Research}
Yes

\subsection{Aim}
to discuss recent and prominent proposed game design elements that demonstrate 
their important characteristics in designing educational games, and to categorize these elements into established fundamental elements of educational game design.

\subsection{Key Focus Areas}
categorizing various existing game elements into established fundamental elements

\subsection{Gaps Addressed}
\begin{itemize}
    \item  The paper addresses the gap in theoretical frameworks like:
    \begin{itemize}
        \item \textbf{the balance framework}, which aims to balance player's skill level with the challenge of the game or aligning the game's realism (fidelity) with educational goals.
        \item \textbf{The Gaming System Framework}: This framework is divided into three levels:
    \begin{itemize}
        \item \textbf{Micro level}: Focuses on problem-solving and challenges that the player faces. It emphasizes learning outcomes through motivational and exploratory experiences, helping players develop skills.
        \item \textbf{Macro level}: Deals with the game's fiction and scenarios, focusing on how players adopt gameplay strategies and enhance their identity. This level aims to support motivation through immersive, experiential learning.
        \item \textbf{Metalevel}: Divided into two sublevels:
        \begin{itemize}
            \item \textbf{Builder level}: Involves contributing to game design skills.
            \item \textbf{Social level}: Focuses on social experiences and the social identity of players.
        \end{itemize}
    \end{itemize}
    \item \textbf{Collaborative Multiplayer Game Framework}: This framework is based on multiplayer game dynamics and design principles. It:
    \begin{itemize}
        \item Begins with analyzing an existing player model to understand the audience.
        \item Proposes a typology of gameplay themes to help designers visualize actions within the game.
        \item Outlines five key components for game design: learning objectives, story, 3D world, gameplay, and evaluation. These elements help guide the design process to ensure that the game aligns with educational and gameplay goals.
    \end{itemize}
    \end{itemize}
\end{itemize}

\subsection{Findings}
\begin{itemize}
    \item Elemental Tetrad is a model used in many games in the modern-day.
    \item Elemental Pentad is a model that describes the five key components of a game: mechanics, story, aesthetics, technology, and education.
    \item Educational game design elements can be categorized into established fundamental elements.
    \item The Elemental Pentad can be used to categorize game design elements into educational game design fundamentals.
\end{itemize}


\subsection{Future Work}
\begin{itemize}
    \item Ambiguity and duplication in game mechanics hinder clarity.
    \item Repetition of design elements (challenges) under the story fundamental element leads to confusion.
    \item Confusion arises from general vs. specific concepts in educational game design.
    \item A deeper understanding is needed to address the duplication of proposed game elements and the limited use of the Elemental Pentad.
    \item There is a need to categorize terminology used for educational games (e.g., serious games, effective video games).
    \item The rapid emergence of game design elements leads to duplication, necessitating clearer definitions (e.g., elements, factors, key elements).
    \item A need exists to reduce duplication of design elements to avoid confusion among researchers and practitioners.
    \item Establishing a common language may assist game designers in communication and organization.
    \item Stakeholders (educators, teachers, learners) should relate their experiences to design elements to facilitate the game design process.
\end{itemize}
    

\section{Notes}
\begin{itemize}
\item the Elemental Tetrad is a model that describes the four key components of a game: mechanics, story, aesthetics, and technology.

\end{itemize}


\newpage
\section{VoRtex Metaverse Platform for Gamified Collaborative Learning}

\subsection{Aim}
The paper introduces a platform called VoRtex, designed to offer tools for creating educational experiences in virtual worlds, especially during pandemic situations.

\subsection{Key Focus Areas}
\begin{itemize}
    \item Software architecture and tools for the VoRtex platform.
    \item Collaborative learning within a virtual environment.
    \item Educational experiences designed for pandemic situations.
\end{itemize}

\newpage
\section{Integrating Generative AI in Hackathons: Opportunities, Challenges, and Educational Implications}

\subsection{Aim}
This study explores the impact of generative AI on students' technological choices, focusing on a case study from the University of Iowa's 2023 event.

\subsection{Key Focus Areas}
\begin{itemize}
    \item Impact of generative AI on technological choices in hackathons.
    \item Educational implications of integrating AI in student-led events.
    \item Balancing innovation with ethical considerations in educational environments.
\end{itemize}

\newpage
\section{Development of Gamification Model for Personalized E-Learning}

\subsection{Aim}
This study aims to design, implement, and evaluate a personality-based gamification model for e-learning systems, enhancing personalization in learning environments.

\subsection{Key Focus Areas}
\begin{itemize}
    \item Personalization in e-learning through gamification based on MBTI.
    \item Engagement metrics such as appeal, emotion, user-centricity, and satisfaction.
    \item Educational usability criteria like clarity, error correction, and feedback.
\end{itemize}

\newpage
\section{Scribbles and brainstorming on the side}

\subsection{Questions to Explore and reflect on when reading}
\begin{itemize}
    \item learning is about trying new things and failing, so how can we gamify that?
    \item what we like about games is the sense of progression and achievement, how can we bring that to learning?
    \item games give the player room to explore and make choices which affect the outcome but also allow them to fail and try again, how can we bring that to learning?
    \item why is an app like duolingo so successful in teaching languages? what can we learn from that?
    \item no one one's to actually fail, so how can we make failing fun?
    \item also what if we had diffuculty levels where each level is a different learning curve and each diffuculty level gives out different rewards? (we are all winners)
\end{itemize}

\subsection{Game Ideas}
\begin{itemize}
    \item A game where you go up against an AI chatbot that asks you questions and you have to answer them correctly to win points, 
    in return you get to ask the chatbot questions and it will answer them for you. You can call out the chatbot if it gets a question wrong or right which will give you extra points. 
    \textbf{Problem:} How can we make this generative for the Instructor? Can we make it so that the instructor can create "Personalties" for the chatbot to have? 
    like in the paper above where they used Batman and Wednesday Addams as personalities for the learning materials.
    \item rpg game where u use programming to solve puzzles and fight monsters
\end{itemize}

\end{document}
