\chapter{Introduction}\label{chap:intro}

Gamification and the incorporation of game-design elements in non-game contexts has gained considerable momentum in educational settings. The potential it has to raise student motivation, engagement, and improve academic performance has positioned it as one of the most well-known strategies among educators. By incorporating elements of points, badges, leaderboards, and challenges, gamification converts the monotony of learning into an enjoyable process that ultimately renders students proactive in learning.

The issue arises when an instructor needs to create a gamified learning experience for the students, and lacks the skills to create a game on the subject that is being taught. The instructor would need to learn game design and programming to be able to create such game, and even assuming that the instructor in specialized within the field of game design and programming, the time and effort needed to generate a game would be massive and impractical. This is where the tool presented in this paper comes in to play, the tool allows the instructor to generate a game without the need of learning programming or game design, and only needs the content that would be used within the game itself. The tool is designed to be educational and ensures gamified elements are present within the game. 

A number studies have underlined the effectiveness of gamification in educational fields. For instance, the work by Lara-Cabrera et al. (2023) has shown that gamified techniques developed with 3D-printed badges had positive influences on performance but also reduced the rate of STEM programs dropouts \cite{lara2023badges}. Jack et al. (2024) assessed the efficiency of using gamification within flipped classroom learning environments and reported significant increases in students' engagement and motivation based on active learning efforts \cite{jack2024gamification}. Gameification and game-based learning follow game design approaches and frameworks to create a more engaging and interactive learning experience.
Game-Based Learning-Game-Based Approach: Another angle that has proved to show positive results includes the Game-Based Learning process. Fernando et al. 2024 have accentuated the role of GBL for fostering student motivation through immediate feedback, clear objectives, and a sense of accomplishment. Thus, these findings have come to demonstrate the ever-changing potential of gamification in education; they provide proof that gamification can create meaningful and engaging learning experiences beyond those possible through conventional pedagogies. Personalized gamification has also emerged as an exciting area of study, in which a procedurally generated system can create tailored goals and challenges for players based on their performance. The instructors have been also a great focus of gamification studies, as they could control what is being offered to the students, however, the tools available for them are still limited. For instance, The study by Gordillo et al. (2020) introduced a game-based learning platform that allows educators to create and share educational games with their students. The platform features a library of pre-built games and educational content covering various subjects and grade levels, making it easy for teachers to implement engaging learning activities in their classrooms \cite{sgame2020}.

Reza et al. presented an automated system that automatically generated customized challenges in accordance with the preference, progress, and performance of the learners to overcome some limitations found in already developed gamified systems \\cite{khoshkangini2021}. Setting off these remarkable benefits, on the other hand, gamification is still an evolving area of study that boasts a number of challenges yet to be overcome. Probably one important limitation is that easy-to-use tools for educators who want to create gamified learning experiences without technical training in programming or game design are still in their infancy. Most of these systems are based on expert knowledge, making it hard for instructors who can't afford the resources or expertise necessary to implement gamified strategies themselves. User-centered tools in this respect could realistically let more educators take advantage of gamification, with further potentials to address a wide range from personalized and active learning settings.