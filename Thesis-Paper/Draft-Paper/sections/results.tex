\chapter{Experiments \& Results}\label{chap:results}

\section{Experimental Setup}

Experiments with educators were carried out to comprehend and assess the suggested tool's usefulness and efficacy. The experimental setting is described in depth in the parts that follow, along with subject recruitment, data collection techniques, assessment measures, and experiment results. The tests were divided into two stages: a usability study and a focus group session. In contrast to the usability study, which involved individual educators interacting with the platform to complete specific tasks and providing feedback via a questionnaire, the focus group session involved a group of educators who volunteered to provide feedback on the platform's features, interface, and overall usability after testing the platform.


\section{Focus Group Session}

Five teaching assistants from the German International University (GIU) with bachelor's degrees or above participated in the focus group. Volunteers from various academic departments and specialties participated, including two instructors from the Computer Science department, one from the Business department, and two from the Business Informatics department. Following a presentation and explanation of the InstaGame, attendees were asked to test out the platform and offer input on its features, usability, and possible applications in their classes. Thirty minutes were spent in the focus group. Prior to the usability assessment, the platform's functionality and design were improved based on input gathered from the focus group. Suggestions for enhancing the user interface for easier navigation and a better user experience overall were included in the feedback. The platform's potential uses in their courses and ways to improve student motivation and engagement were also discussed by the participants. One teacher assistant from the business department proposed using the tool to create a case study game in a talent acquisition setting so that students could put what they had learned in lectures into practice.


\section{Usability Study}
