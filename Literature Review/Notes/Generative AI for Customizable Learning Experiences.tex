\section{Generative AI for Customizable Learning Experiences}

\subsection{Authors}
\begin{itemize}
    \item \textbf{Authors:} Ivica Pesovski, Ricardo Santos, Roberto Henriques, Vladimir Trajkovik
\end{itemize}

\subsection{Publication}
\begin{itemize}
    \item \textbf{Publication:} Sustainability, Volume 16, Article 3034
    \item \textbf{Date:} April 2024
    \item \textbf{DOI:} \href{https://doi.org/10.3390/su16073034}{10.3390/su16073034}
\end{itemize}

\subsection{Relevant to My Research}
Loosely relevant

\subsection{Aim}
The paper proposes an affordable and sustainable approach to personalizing learning materials, developing a tool integrated into an LMS.

\subsection{Key Focus Areas}
\begin{itemize}
    \item Generative AI and Personalized Learning.
    \item Tool integrated into LMS for generating materials based on learning outcomes.
    \item Learning materials in three formats: traditional, pop-culture-inspired (e.g., Batman, Wednesday Addams).
    \item Assessment via multiple-choice questions.
    \item Experiment with 20 software engineering students.
    \item Positive findings: engagement and increased study time.
\end{itemize}

\subsection{Gaps Addressed}
\begin{itemize}
    \item Lack of empirical studies on LLMs in classrooms.
    \item Limited research on AI-generated lesson creation.
    \item Unclear effectiveness of virtual AI instructors.
\end{itemize}

\subsection{Findings}
\begin{itemize}
    \item Students found the different styles of learning materials engaging. While the traditional style was most used, pop-culture-inspired materials doubled the study time. AI-generated quizzes were also popular for self-assessment.
    
    \item The variety of formats (traditional and pop-culture inspired) increased engagement, particularly for students struggling with the material. Initially, fictional styles were preferred, but traditional materials became the favorite for long-term learning six months after the course.
    
    \item The experiment showed that AI-generated materials benefited less proficient students and promoted inclusivity. The method can be applied across various educational contexts to offer personalized and effective learning experiences.
\end{itemize}


\subsection{Future Work}
\begin{itemize}
    \item Further study on integrating LLMs into more educational contexts.
    \item Long-term research on the reception of AI-generated content.
\end{itemize}

\subsection{Notes}
The paper emphasizes that generative AI can engage students through personalized, multi-format content that caters to various learning styles.
