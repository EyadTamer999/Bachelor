\chapter{Conclusion}\label{chap:concl}
InstaGame is a means to support the instructor and the students in their educational experience and eases the search for games that matches the conent that the instructor needs to teach, as the platform supports various game templates to be used in various fields, and the unneeded programming or game design experience to generate a game for the students. Also InstaGame supports being played on any browser and eases the sharing of the game through multiple means of distribution. The platform was tested with educators and the impressions were mainly postive towards the tool as it was seen as a tool that would ease the teaching experience and make it more engaging for the students. 
 
\section{Future work}

The work presented can imporve and grow in many aspects, more quantitative testing needs to be done to ensure that the platform is useful and engaging for the students, as it was only tested on instructors. More game templates that allow for more immersive experiences, and games that support co-op multiplayer to teach students more about teamwork. Future work should also consider the creative process of creating games and the game design elements. The tool presented can also be used in other fields such as corporate training, and the platform can be used to create games for team building exercises. Data collection and a dashboard for the instructors to use to analyze the students performance and engagement in the games, should be considered in future work. Accessibility should also be considered in future work, to ensure that students with disabilities can also use the platform, the unity engine supports accessibility features that can be used to make the games more accessible by remapping the controls or changing the color scheme of the game. It should be also considered to add more structural gameification features such as leaderboards, badges, and achievements to make the games more engaging for the students and keep them returning to the platform.

\section{Limitations}

The limitations of the work presented makes room for improvements and more testing of InstaGame, as the platform was only tested with five educators in a focus group setting. The platform also lacks the amount of game templates as the there exists only two templates to be chosen from, and those templates contain minor bugs, but more bugs may remain undiscovered. The platform also does not show if the student has completed the task or not, and there is no any data collection for anaylsis that the instructors to use. Also in order to generate a game there has to exist a game templatye that the instructor can use, therefore the platform does not generate games from scratch. The use of structural gameification features such as leaderboards, badges, and achievements are also missing from the platform.