\chapter{Experiments \& Results}\label{chap:results}

\section{Experimental Setup}
To evaluate the effectiveness of the proposed system, a series of experiments with educators was conducted. The experiments aimed to assess the usability, functionality, and impact of the platform on learning outcomes. The following sections detail the experimental setup, including participant recruitment, data collection methods, and evaluation metrics.

the experiments were conducted in two phases: a focus group session and a usability study. The focus group session involved a group of educators who provided feedback on the platform's features, interface, and overall usability, the usability study, on the other hand, involved individual educators who interacted with the platform to complete specific tasks and provide feedback on their experience.

The focus group session was conducted with a group of 5 educators from the German International University (GIU). The participants were volunteers from various faculties, 2 instructors from the Computer Science department, 1 from the Business department, and 2 from the Business Informatics department. The InstaGame was presented and explained to the participants, then they were asked to try out the platform and provide feedback on its features, usability, and potential use cases in their courses. The focus group session lasted for 30 minutes. The feedback collected from the focus group session was used to refine the platform's design and functionality before the usability study. The feedback included suggestions for improving the user interface for better navigation and an overall better user experience. The participants also provided insights into the platform's potential applications in their courses and how it could enhance student engagement and motivation.

