\chapter{Related Work}\label{chap:relatedwork}
The fundamental ideas of gamification, its uses in educational contexts, and its prior research in the area of game-based learning are covered in this chapter. It also talks about the possible advantages and difficulties of gamification in the classroom, emphasizing the necessity for easily accessible resources that let teachers design customized gamified lessons.

Gamification can be categorized into three primary types: content-based gamification, structural gamification, and game-based learning (GBL). Content-based gamification involves embedding game-like elements into existing educational content, such as quizzes, assignments, or lectures, to make learning more interactive and engaging. In contrast, structural gamification focuses on redesigning the overall learning environment by incorporating game-like mechanics, such as progress tracking, rewards, or challenges, to enhance student motivation and engagement \cite{fernando2024}.

Game-based learning (GBL) has been proven as an effecetive means of education and is equally as effecetive as traditional learning methods, as Gordillo et al. (2020) have shown \cite{sgame2020}. Another reaserch introduced a 3D game authoring tool for users with little to no programming knowledge using OpenAi's GPT-4, however the toolkit has not been tested by instructors or students \cite{horn2023}. The use of structural gamification was also used to make visualization of data more engaging and interactive, as shown by the work of Karuna et al. (2022), as The study explores using gamification elements to enhance user engagement in data collection, focusing on game design and behavioral data visualization, the study later  concluded that using gamification in data collection, especially through visual and interactive methods, can significantly enhance user engagement and data quality. \cite{karuna2019}.


As outlined by Fernando and Premadasa (2024), these approaches are central to understanding how gamification and GBL can be effectively employed in educating Generation Alpha. Their systematic literature review highlights how these methods influence learning outcomes by promoting active participation and fostering deeper engagement in educational settings \cite{fernando2024}. The procedural generation of challenges and content in GBL has also been explored by Khoshkangini et al. (2021), who developed an automated system that generates personalized challenges based on player preferences, progress, and performance. This system addresses existing limitations in gamified learning environments by providing tailored experiences that cater to individual needs and learning styles \cite{khoshkangini2021}. Another personalized gamification system was proposed by Noor et al. (2010), which explores the automatic generation of personalized levels for platform games, which uses procedural content generation to create unique game levels based on player preferences and performance \cite{noor2010}. None of these systems, however, provide educators with the tools to design their own gamified learning experiences, limiting their potential impact on classroom instruction.

In \emph{The Art of Computer Game Design}, Chris Crawford describes a game as an interactive medium where players make choices and experience the consequences of those choices. This definition is intentionally broad, encompassing various forms of games, such as board games, card games, sports, video games, and even educational games. Educational games, in particular, are designed to teach players about specific subjects, reinforce concepts, facilitate skill development, or help them understand historical events or cultures through gameplay \cite{crawford1982art}.

One of the closest works related to the work presented is the study by Gordillo et al. (2020) \cite{sgame2020}, which introduces a game-based learning platform designed for educators to create and share educational games with their students. The platform features a user-friendly interface that allows educators to design games, integrate multimedia content, and monitor student progress. Additionally, it includes a library of pre-built games covering various subjects and grade levels, making it easy for teachers to implement engaging learning activities in their classrooms. However, while Gordillo et al.’s work primarily focuses on game-based learning with preset game templates and pre-defined educational content, this thesis takes a different approach. It seeks to offer a more flexible and customizable platform, enabling educators to design gamified learning experiences tailored specifically to their unique needs and teaching preferences.