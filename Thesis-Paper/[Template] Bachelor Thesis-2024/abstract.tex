\chapter*{Abstract}
% \addcontentsline{toc}{chapter}{Abstract}
\label{chap:abstract}


The proliferation of image forgery in the digital age has raised concerns regarding the authenticity and credibility of visual information. This thesis focuses on the detection of forgery in digital images, specifically image splicing. Image splicing is a method of forgery where different sections of multiple images are combined to create a composite image. The thesis proposes two deep vision models: the ELA-DenseNet for image forgery classification and a fine-tuned vision transformer for the localization of manipulated areas. The importance of data quality and quantity is emphasized by introducing an enriched dataset specifically curated and annotated for image splicing detection. The results of the classification model outperform other similar state of the art models and the localization model has shown promising results in tandem with the proposed dataset, thereby advancing the capabilities of forgery detection in this domain and aiding in preserving the trust and integrity of visual information in an era of widespread disinformation.

