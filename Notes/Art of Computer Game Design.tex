\section{Art of Computer Game Design}

\subsection{Authors}
\begin{itemize}
    \item \textbf{Author:} Chris Crawford
\end{itemize}

\subsection{Publication}
\begin{itemize}
    \item \textbf{Publication:} Washington State University
    \item \textbf{Date:} Originally published in 1982, with an electronic version in 1997
    \item \textbf{DOI:} Not available; however, the electronic version is accessible at \href{http://www.vancouver.wsu.edu/fac/peabody/game-book/Coverpage.html}{Washington State University Vancouver}
\end{itemize}

\subsection{Relevant to My Research}
Highly relevant for understanding the fundamental principles of game design and the role of games as an art form.

\subsection{Aim}
The book explores the concept that computer games are a new art form, focusing on how games evoke emotion through fantasy, and discusses various aspects of game design, including representation, interaction, conflict, and safety.

\subsection{Key Focus Areas}
\begin{itemize}
    \item Definition of games and their fundamental components: representation, interaction, conflict, and safety.
    \item The role of fantasy and participation in games, making them an interactive art form.
    \item A comparison of games with other art forms like music and literature, emphasizing the participatory nature of games.
    \item Game design principles and techniques.
    \item The importance of emotional engagement in games.
    \item A taxonomy of computer games, including skill-and-action games, strategy games, and more.
\end{itemize}

\subsection{Gaps Addressed}
\begin{itemize}
    \item Lack of established theories on game design.
    \item Minimal focus on the artistic potential of computer games in the early 1980s.
    \item The book challenges the view that computer games are mere entertainment, emphasizing their capacity as an art form.
\end{itemize}

\subsection{Findings}
\begin{itemize}
    \item Games are an interactive form of art, allowing players to create their own experiences within the framework provided by the designer.
    \item The participatory nature of games offers a deeper emotional engagement compared to other art forms.
    \item Computer games, though still developing as an art form in the early 1980s, hold vast potential for emotional and intellectual engagement.
    \item The success of game design depends on understanding the fundamental elements of games and creating experiences that resonate emotionally with players.
\end{itemize}

\subsection{Future Work}
\begin{itemize}
    \item Further exploration of games as an art form.
    \item Development of more sophisticated game design theories and methodologies.
    \item Expansion of the taxonomy of computer games as the industry evolves.
\end{itemize}

\subsection{Notes}
This foundational book in the field of game design emphasizes the artistic and participatory aspects of games, proposing that computer games have the potential to evoke deep emotional responses through interactive fantasy worlds. The text provides a basis for understanding the core principles of game design and their artistic potential.
