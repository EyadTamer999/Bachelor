\chapter{Introduction}
\label{chap:intro}

The role of images in human communication has evolved significantly, especially with the advent of Web 2.0, which transformed online experiences to be more interactive and participatory \cite{Murugesan_2007}. Web 2.0 describes the current state of the internet that focuses on user generated content and participatory culture. This shift allowed for the efficient dissemination of visual information, empowering individuals and spearheading the era of citizen journalism \cite{citizenJournalism}. In this new era, amateur contributors gain the ability to amass exposure and influence comparable to that of more established organizations. While it can mostly be positive, this democratization of image dissemination brought about a concerning issue – the ease of forging information, particularly images. Advanced image processing software such as Photoshop, Corel Paint Shop and GIMP has made it easier than ever to create deceptive visual content, leading to an increase in image forgery and contributing to the current era of disinformation. In this context, disinformation refers to intentionally spreading false information, distinct from misinformation, which is false but is not shared with malicious intent. \cite{MuhammedT_Mathew_2022}

The consequences of image forgery extend beyond personal narratives and social media, affecting security and legal contexts where the authenticity of visual evidence is crucial. Manipulated images can be used to construct false narratives, present falsified evidence, or deceive viewers, compromising the credibility of information and distorting public perceptions. Recognizing the gravity of disinformation, particularly emanating from falsified images, the United Nations has identified it as a phenomenon requiring concerted efforts to combat \cite{UNReport}, reflecting the concern of 85\% of people worldwide, as revealed by a 2023 global survey. \cite{Henley_2023}


Addressing the challenges of image forgery-induced disinformation involves the application of digital forensics techniques. Digital forensics, a branch of forensic science, involves collecting, analyzing, and preserving digital evidence to investigate and establish the authenticity and integrity of digital artifacts \cite{zakariah2018}. In image forensics, digital forensics methodologies and tools are used to detect and analyze signs of image manipulation. Leveraging advancements in computer vision, pattern recognition, and machine learning, digital forensics experts can develop robust algorithms to identify forged images, thereby preserving trust and integrity in visual information. \cite{FERREIRA2020106685} 

This thesis focuses on a crucial facet of image forgery detection - image splicing detection. Image splicing is a forgery method that involves artfully combining different sections of multiple images to craft a composite image, typically through cutting and pasting \cite{Basavaraj2022}. The detection process involves two primary steps. The initial step is a classification task dedicated to determining the authenticity of the image. The second step, termed localization, entails locating the `forged' portion of the image or predicting its mask—a black-and-white image that designates the tampered area. An illustrative example of an image mask is presented in Figure \ref{fig:imgMask}. In tandem, this work's central contribution in the field of image splicing detection arises from the collaborative efforts of two distinct models. One model, the ELA-DenseNet, is dedicated to the classification task, discerning the authenticity of the image through the integration of Error Level Analysis (ELA) and DenseNet architecture. The second model, a fine-tuned vision transformer, specializes in the crucial task of localizing manipulated areas within the image. Together, these two models form a comprehensive approach to address the challenges posed by image splicing, collectively advancing the capabilities of forgery detection in this domain.

In addition to the two models, the work extends the purview of image splicing detection by emphasizing the significance of data quality and quantity. A meticulously expanded dataset tailored for image splicing detection is introduced, encompassing a diverse array of manipulated images. This enriched dataset serves as a valuable resource for training and evaluating the proposed vision transformer and other models, ultimately enhancing generalization and performance in real-world scenarios. The collective efforts of the vision transformer, ELA-DenseNet model, and the enriched dataset collectively propel the capabilities of image forgery detection, particularly in the domain of image splicing.



\begin{figure}[!h]
  \centering
  \includegraphics[width=0.8\linewidth]{figures/ImageMask.png} % Replace "example-image" with the filename of your image
  \caption{An image alongside its image mask showing the tampered region in white \cite{pandey2022detecting}.}
  \label{fig:imgMask}
\end{figure}

