\documentclass{article}
\usepackage{hyperref}
\usepackage{amsmath}
\usepackage[utf8]{inputenc}

\title{Space Invaders Notes}
\author{Eyad Hassan}
\date{\today}

\begin{document}

\maketitle

\section{Introduction}
This document contains Game Notes


\newpage

\section{Game Flow}

\begin{itemize}
    \item Instructor:
    \begin{itemize}
        \item \textbf{Chooses number of levels:}
        \begin{itemize}
            \item \textbf{Chooses Static Fields:} ex: "Convert to Binary: " and chooses type of challenge: Order/Last killed
            \item \textbf{Chooses Dynamic Fields:} 
            \begin{itemize}
                \item Enters a range to choose from randomly ex: 0 to 15
                \item Enters the number of turns or sublevels within that level (Should be an odd number)
            \end{itemize}
            \item Submit and move on to the next level if there is a next level
            \item If there are no more levels to show, then show a panel with a summary to each level if the instructor wishes to edit any of the levels
        \end{itemize}
    \end{itemize}
\end{itemize}

\newpage

\section{Challenge Types}
This section explains the challenge types within the game.
A level is won if the number of turns or sublevels is greater than the number of lost levels
\begin{itemize}
    \item Order: The answer is inputed based on the order the enemies were killed in.
    \item Last Killed: The Game Manager checks on the last killed enemy and compares it to the correct answer.
\end{itemize}

\newpage

\section{Bugs and Issues}

\newpage
\section{Links}
\begin{itemize}
    \item \href{https://www.figma.com/design/LqO6EM0LpA8Q7YLeUj8eEN/Bachelor?node-id=0-1&t=Ogt33dzuluNJpiJL-1}{Figma}
\end{itemize}

\end{document}
