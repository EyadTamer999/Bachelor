\chapter{Conclusion}\label{chap:concl}

In conclusion, this thesis addresses the pressing issue of image forgery and its impact on the authenticity and integrity of visual information. By focusing on image splicing detection, the thesis provides insights into the classification and localization of manipulated areas within images. The proposed models, the ELA-DenseNet and the fine-tuned vision transformer, demonstrate promising results in detecting image splicing, contributing to the field of digital forensics. Additionally, the enriched dataset enhanced the generalization and performance of the localization model and provides good representative data for forgeries encountered in realistic scenarios online. The findings of this thesis highlight the importance of robust algorithms and quality datasets in combating image forgery-induced disinformation. By preserving trust and integrity in visual information, the proposed methods contribute to the mitigation of the negative consequences of image forgery in various contexts, including security, legal, and social domains.

This thesis opens up several avenues for future research in the field of image forgery detection. Firstly, the proposed models can be further optimized and fine-tuned to improve their performance and generalizability. Exploring different network architectures, incorporating additional features, or utilizing advanced optimization techniques could potentially enhance the accuracy and robustness of the models. Secondly, the research can be extended to develop real-time forgery detection systems that can process images in a streaming fashion, enabling the detection of manipulated content in near real-time scenarios. Finally, as image forgery techniques continue to evolve, it is crucial to continuously update and expand the dataset used for evaluation to ensure the effectiveness of the proposed models in detecting new and sophisticated forms of forgery. By addressing these future directions, researchers can further advance the field of image forgery detection and contribute to the development of reliable and robust techniques for preserving the integrity of visual information.

