\section{Categorizing Game Design Elements into Educational Game Design Fundamentals}

\subsection{Authors}
\begin{itemize}
    \item \textbf{Authors:} Mifrah Ahmad
\end{itemize}

\subsection{Publication}
\begin{itemize}
    \item \textbf{Publication:}  10.5772/intechopen.89971
    \item \textbf{Date:} 20 November 2019
\end{itemize}

\subsection{Relevant to My Research}
Yes

\subsection{Aim}
to discuss recent and prominent proposed game design elements that demonstrate 
their important characteristics in designing educational games, and to categorize these elements into established fundamental elements of educational game design.

\subsection{Key Focus Areas}
categorizing various existing game elements into established fundamental elements

\subsection{Gaps Addressed}
\begin{itemize}
    \item  The paper addresses the gap in theoretical frameworks like:
    \begin{itemize}
        \item \textbf{the balance framework}, which aims to balance player's skill level with the challenge of the game or aligning the game's realism (fidelity) with educational goals.
        \item \textbf{The Gaming System Framework}: This framework is divided into three levels:
    \begin{itemize}
        \item \textbf{Micro level}: Focuses on problem-solving and challenges that the player faces. It emphasizes learning outcomes through motivational and exploratory experiences, helping players develop skills.
        \item \textbf{Macro level}: Deals with the game's fiction and scenarios, focusing on how players adopt gameplay strategies and enhance their identity. This level aims to support motivation through immersive, experiential learning.
        \item \textbf{Metalevel}: Divided into two sublevels:
        \begin{itemize}
            \item \textbf{Builder level}: Involves contributing to game design skills.
            \item \textbf{Social level}: Focuses on social experiences and the social identity of players.
        \end{itemize}
    \end{itemize}
    \item \textbf{Collaborative Multiplayer Game Framework}: This framework is based on multiplayer game dynamics and design principles. It:
    \begin{itemize}
        \item Begins with analyzing an existing player model to understand the audience.
        \item Proposes a typology of gameplay themes to help designers visualize actions within the game.
        \item Outlines five key components for game design: learning objectives, story, 3D world, gameplay, and evaluation. These elements help guide the design process to ensure that the game aligns with educational and gameplay goals.
    \end{itemize}
    \end{itemize}
\end{itemize}

\subsection{Findings}
\begin{itemize}
    \item Elemental Tetrad is a model used in many games in the modern-day.
    \item Elemental Pentad is a model that describes the five key components of a game: mechanics, story, aesthetics, technology, and education.
    \item Educational game design elements can be categorized into established fundamental elements.
    \item The Elemental Pentad can be used to categorize game design elements into educational game design fundamentals.
\end{itemize}


\subsection{Future Work}
\begin{itemize}
    \item Ambiguity and duplication in game mechanics hinder clarity.
    \item Repetition of design elements (challenges) under the story fundamental element leads to confusion.
    \item Confusion arises from general vs. specific concepts in educational game design.
    \item A deeper understanding is needed to address the duplication of proposed game elements and the limited use of the Elemental Pentad.
    \item There is a need to categorize terminology used for educational games (e.g., serious games, effective video games).
    \item The rapid emergence of game design elements leads to duplication, necessitating clearer definitions (e.g., elements, factors, key elements).
    \item A need exists to reduce duplication of design elements to avoid confusion among researchers and practitioners.
    \item Establishing a common language may assist game designers in communication and organization.
    \item Stakeholders (educators, teachers, learners) should relate their experiences to design elements to facilitate the game design process.
\end{itemize}
    

\section{Notes}
\begin{itemize}
\item the Elemental Tetrad is a model that describes the four key components of a game: mechanics, story, aesthetics, and technology.

\end{itemize}
