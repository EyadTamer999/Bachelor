\chapter{Background}\label{chap:background}
This chapter outlines the core concepts of gamification, its applications in educational settings, and its past work in the field of game-based learning. It also discusses the potential benefits and challenges associated with gamification in education, highlighting the need for accessible tools that enable educators to create personalized gamified learning experiences.

Gamification can be categorized into three primary types: content-based gamification, structural gamification, and game-based learning (GBL). Content-based gamification involves embedding game-like elements into existing educational content, such as quizzes, assignments, or lectures, to make learning more interactive and engaging. In contrast, structural gamification focuses on redesigning the overall learning environment by incorporating game-like mechanics, such as progress tracking, rewards, or challenges, to enhance student motivation and engagement \cite{fernando2024}.

As outlined by Fernando and Premadasa (2024), these approaches are central to understanding how gamification and GBL can be effectively employed in educating Generation Alpha. Their systematic literature review highlights how these methods influence learning outcomes by promoting active participation and fostering deeper engagement in educational settings \cite{fernando2024}. The procedural generation of challenges and content in GBL has also been explored by Khoshkangini et al. (2021), who developed an automated system that generates personalized challenges based on player preferences, progress, and performance. This system addresses existing limitations in gamified learning environments by providing tailored experiences that cater to individual needs and learning styles \cite{khoshkangini2021}. Another personalized gamification system was proposed by Noor et al. (2010), which explores the automatic generation of personalized levels for platform games, which uses procedural content generation to create unique game levels based on player preferences and performance \cite{noor2010}. None of these systems, however, provide educators with the tools to design their own gamified learning experiences, limiting their potential impact on classroom instruction.

In \emph{The Art of Computer Game Design}, Chris Crawford describes a game as an interactive medium where players make choices and experience the consequences of those choices. This definition is intentionally broad, encompassing various forms of games, such as board games, card games, sports, video games, and even educational games. Educational games, in particular, are designed to teach players about specific subjects, reinforce concepts, facilitate skill development, or help them understand historical events or cultures through gameplay \cite{crawford1982art}.

One of the closest works related to this thesis is the study by Gordillo et al. (2020) \cite{sgame2020}, which introduces a game-based learning platform designed for educators to create and share educational games with their students. The platform features a user-friendly interface that allows educators to design games, integrate multimedia content, and monitor student progress. Additionally, it includes a library of pre-built games covering various subjects and grade levels, making it easy for teachers to implement engaging learning activities in their classrooms. However, while Gordillo et al.’s work primarily focuses on game-based learning with preset game templates and pre-defined educational content, this thesis takes a different approach. It seeks to offer a more flexible and customizable platform, enabling educators to design gamified learning experiences tailored specifically to their unique needs and teaching preferences.