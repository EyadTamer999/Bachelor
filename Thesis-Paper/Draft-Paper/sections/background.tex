\chapter{Background}\label{chap:background}
The generation of games started as game developers created games for entertainment purposes. However, the idea of using games for educational purposes has been around for a long time. Educational games are designed to teach players about specific subjects, reinforce concepts, facilitate skill development, or help them understand historical events or cultures through gameplay \cite{crawford1982art}. Educational games can be used in various educational settings, such as schools, universities, and training programs, to engage students and enhance their learning experience. By combining educational content with game mechanics, educators can create interactive and engaging learning activities that motivate students to learn and retain information effectively. Developing a game is a process that requires many skills and resources, including programming knowledge, game design expertise, and multimedia content creation. This can be a significant barrier for educators who lack the technical skills or resources to create their own educational games. To address this challenge, game-based learning platforms have been developed to provide educators with the tools and resources they need to design and implement educational games in their classrooms. Gamification is a proven means of education that has been show to improve the morale of the user through challenges, rewards, and competition. 

